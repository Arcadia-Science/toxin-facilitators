% Options for packages loaded elsewhere
\PassOptionsToPackage{unicode}{hyperref}
\PassOptionsToPackage{hyphens}{url}
%
\documentclass[
]{article}
\usepackage{amsmath,amssymb}
\usepackage{lmodern}
\usepackage{iftex}
\ifPDFTeX
  \usepackage[T1]{fontenc}
  \usepackage[utf8]{inputenc}
  \usepackage{textcomp} % provide euro and other symbols
\else % if luatex or xetex
  \usepackage{unicode-math}
  \defaultfontfeatures{Scale=MatchLowercase}
  \defaultfontfeatures[\rmfamily]{Ligatures=TeX,Scale=1}
\fi
% Use upquote if available, for straight quotes in verbatim environments
\IfFileExists{upquote.sty}{\usepackage{upquote}}{}
\IfFileExists{microtype.sty}{% use microtype if available
  \usepackage[]{microtype}
  \UseMicrotypeSet[protrusion]{basicmath} % disable protrusion for tt fonts
}{}
\makeatletter
\@ifundefined{KOMAClassName}{% if non-KOMA class
  \IfFileExists{parskip.sty}{%
    \usepackage{parskip}
  }{% else
    \setlength{\parindent}{0pt}
    \setlength{\parskip}{6pt plus 2pt minus 1pt}}
}{% if KOMA class
  \KOMAoptions{parskip=half}}
\makeatother
\usepackage{xcolor}
\usepackage[margin=1in]{geometry}
\usepackage{color}
\usepackage{fancyvrb}
\newcommand{\VerbBar}{|}
\newcommand{\VERB}{\Verb[commandchars=\\\{\}]}
\DefineVerbatimEnvironment{Highlighting}{Verbatim}{commandchars=\\\{\}}
% Add ',fontsize=\small' for more characters per line
\usepackage{framed}
\definecolor{shadecolor}{RGB}{248,248,248}
\newenvironment{Shaded}{\begin{snugshade}}{\end{snugshade}}
\newcommand{\AlertTok}[1]{\textcolor[rgb]{0.94,0.16,0.16}{#1}}
\newcommand{\AnnotationTok}[1]{\textcolor[rgb]{0.56,0.35,0.01}{\textbf{\textit{#1}}}}
\newcommand{\AttributeTok}[1]{\textcolor[rgb]{0.77,0.63,0.00}{#1}}
\newcommand{\BaseNTok}[1]{\textcolor[rgb]{0.00,0.00,0.81}{#1}}
\newcommand{\BuiltInTok}[1]{#1}
\newcommand{\CharTok}[1]{\textcolor[rgb]{0.31,0.60,0.02}{#1}}
\newcommand{\CommentTok}[1]{\textcolor[rgb]{0.56,0.35,0.01}{\textit{#1}}}
\newcommand{\CommentVarTok}[1]{\textcolor[rgb]{0.56,0.35,0.01}{\textbf{\textit{#1}}}}
\newcommand{\ConstantTok}[1]{\textcolor[rgb]{0.00,0.00,0.00}{#1}}
\newcommand{\ControlFlowTok}[1]{\textcolor[rgb]{0.13,0.29,0.53}{\textbf{#1}}}
\newcommand{\DataTypeTok}[1]{\textcolor[rgb]{0.13,0.29,0.53}{#1}}
\newcommand{\DecValTok}[1]{\textcolor[rgb]{0.00,0.00,0.81}{#1}}
\newcommand{\DocumentationTok}[1]{\textcolor[rgb]{0.56,0.35,0.01}{\textbf{\textit{#1}}}}
\newcommand{\ErrorTok}[1]{\textcolor[rgb]{0.64,0.00,0.00}{\textbf{#1}}}
\newcommand{\ExtensionTok}[1]{#1}
\newcommand{\FloatTok}[1]{\textcolor[rgb]{0.00,0.00,0.81}{#1}}
\newcommand{\FunctionTok}[1]{\textcolor[rgb]{0.00,0.00,0.00}{#1}}
\newcommand{\ImportTok}[1]{#1}
\newcommand{\InformationTok}[1]{\textcolor[rgb]{0.56,0.35,0.01}{\textbf{\textit{#1}}}}
\newcommand{\KeywordTok}[1]{\textcolor[rgb]{0.13,0.29,0.53}{\textbf{#1}}}
\newcommand{\NormalTok}[1]{#1}
\newcommand{\OperatorTok}[1]{\textcolor[rgb]{0.81,0.36,0.00}{\textbf{#1}}}
\newcommand{\OtherTok}[1]{\textcolor[rgb]{0.56,0.35,0.01}{#1}}
\newcommand{\PreprocessorTok}[1]{\textcolor[rgb]{0.56,0.35,0.01}{\textit{#1}}}
\newcommand{\RegionMarkerTok}[1]{#1}
\newcommand{\SpecialCharTok}[1]{\textcolor[rgb]{0.00,0.00,0.00}{#1}}
\newcommand{\SpecialStringTok}[1]{\textcolor[rgb]{0.31,0.60,0.02}{#1}}
\newcommand{\StringTok}[1]{\textcolor[rgb]{0.31,0.60,0.02}{#1}}
\newcommand{\VariableTok}[1]{\textcolor[rgb]{0.00,0.00,0.00}{#1}}
\newcommand{\VerbatimStringTok}[1]{\textcolor[rgb]{0.31,0.60,0.02}{#1}}
\newcommand{\WarningTok}[1]{\textcolor[rgb]{0.56,0.35,0.01}{\textbf{\textit{#1}}}}
\usepackage{graphicx}
\makeatletter
\def\maxwidth{\ifdim\Gin@nat@width>\linewidth\linewidth\else\Gin@nat@width\fi}
\def\maxheight{\ifdim\Gin@nat@height>\textheight\textheight\else\Gin@nat@height\fi}
\makeatother
% Scale images if necessary, so that they will not overflow the page
% margins by default, and it is still possible to overwrite the defaults
% using explicit options in \includegraphics[width, height, ...]{}
\setkeys{Gin}{width=\maxwidth,height=\maxheight,keepaspectratio}
% Set default figure placement to htbp
\makeatletter
\def\fps@figure{htbp}
\makeatother
\setlength{\emergencystretch}{3em} % prevent overfull lines
\providecommand{\tightlist}{%
  \setlength{\itemsep}{0pt}\setlength{\parskip}{0pt}}
\setcounter{secnumdepth}{-\maxdimen} % remove section numbering
\ifLuaTeX
  \usepackage{selnolig}  % disable illegal ligatures
\fi
\IfFileExists{bookmark.sty}{\usepackage{bookmark}}{\usepackage{hyperref}}
\IfFileExists{xurl.sty}{\usepackage{xurl}}{} % add URL line breaks if available
\urlstyle{same} % disable monospaced font for URLs
\hypersetup{
  pdftitle={Toxin Facilitator Discovery Report},
  hidelinks,
  pdfcreator={LaTeX via pandoc}}

\title{Toxin Facilitator Discovery Report}
\author{}
\date{\vspace{-2.5em}}

\begin{document}
\maketitle

This notebook summarizes the results from the first pass of mining venom
transcriptomes for putative facilitator proteins that aid toxins reach
their final destination. To obtain the results:

\begin{enumerate}
\def\labelenumi{\arabic{enumi}.}
\tightlist
\item
  Accessions from venom transcriptomes were obtained to either directly
  get the proteins or the transcribed RNA sequences to then get
  predicted ORFs/proteins with \texttt{transdecoder}.
\item
  Pool all proteins together and search against toxin Pfam HMMs, keeping
  those that pass a default e-value cutoff of 1e-50
\item
  Annotate all proteins that hit against the toxin Pfam HMMs with all
  Pfam HMMs to get any additional annotations
\end{enumerate}

From these efforts, I was able to collect a total of \textbf{124
transcriptome accessions} resulting in approximately \textbf{300,000
proteins} to search against. Approximately \textbf{5,000 proteins} hit
against the toxin Pfam HMMs, and \textbf{294 proteins} have an
additional Pfam annotation besides the original toxin Pfam annotation
hit.

In the \texttt{results/2023-04-05\_v1} folder, there are several files
to explore the results. The \texttt{summary.txt} describes for each hit
protein the protein length, cysteine content, the Pfam accession and
name that the protein hit against. The
\texttt{all\_additional\_pfam\_annotations.txt} file lists for proteins
that had an additional Pfam annotation besides the original toxin Pfam
hit. The combined toxin protein and protein summary information with the
additional annotation information for the 294 proteins is in the
\texttt{toxin-additional-domain-annotations-hits.csv} file to explore
further.

\begin{Shaded}
\begin{Highlighting}[]
\FunctionTok{library}\NormalTok{(tidyverse)}
\end{Highlighting}
\end{Shaded}

\begin{Shaded}
\begin{Highlighting}[]
\NormalTok{toxin\_summaries }\OtherTok{\textless{}{-}} \FunctionTok{read.table}\NormalTok{(}\StringTok{"../results/2023{-}04{-}05\_v1/summary.txt"}\NormalTok{, }\AttributeTok{header=}\ConstantTok{TRUE}\NormalTok{, }\AttributeTok{sep=}\StringTok{"}\SpecialCharTok{\textbackslash{}t}\StringTok{"}\NormalTok{)}

\NormalTok{additional\_annotations }\OtherTok{\textless{}{-}} \FunctionTok{read.table}\NormalTok{(}\StringTok{"../results/2023{-}04{-}05\_v1/all\_additional\_pfam\_annotations.txt"}\NormalTok{, }\AttributeTok{header=}\ConstantTok{TRUE}\NormalTok{, }\AttributeTok{sep=}\StringTok{"}\SpecialCharTok{\textbackslash{}t}\StringTok{"}\NormalTok{)}

\FunctionTok{head}\NormalTok{(toxin\_summaries)}
\end{Highlighting}
\end{Shaded}

\begin{verbatim}
##                                                               hit_id length
## 1 Bungarus_multicinctus_GIKH01.transdecoder.pep_id_GIKH01024005.1.p1   1352
## 2    Psyttalia_concolor_GCDX01.transdecoder.pep_id_GCDX01003069.1.p1   1247
## 3    Psyttalia_concolor_GCDX01.transdecoder.pep_id_GCDX01009444.1.p1   1324
## 4    Psyttalia_concolor_GCDX01.transdecoder.pep_id_GCDX01004213.1.p1   1345
## 5    Psyttalia_concolor_GCDX01.transdecoder.pep_id_GCDX01012554.1.p1   1345
## 6     Ampulex_compressa_GFCP01.transdecoder.pep_id_GFCP01024109.1.p1   1345
##   cysteine_content pfam_query_accession pfam_query_name
## 1             0.02           PF00005.30        ABC_tran
## 2             0.01           PF00005.30        ABC_tran
## 3             0.01           PF00005.30        ABC_tran
## 4             0.01           PF00005.30        ABC_tran
## 5             0.01           PF00005.30        ABC_tran
## 6             0.01           PF00005.30        ABC_tran
\end{verbatim}

\begin{Shaded}
\begin{Highlighting}[]
\FunctionTok{head}\NormalTok{(additional\_annotations)}
\end{Highlighting}
\end{Shaded}

\begin{verbatim}
##                                                                   hit_id
## 1    Pseudagkistrodon_rudis_GDKV01.transdecoder.pep_id_GDKV01024984.1.p1
## 2 Hemiscolopendra_marginata_GHBY01.transdecoder.pep_id_GHBY01000372.1.p1
## 3     Bungarus_multicinctus_GIKH01.transdecoder.pep_id_GIKH01029928.1.p1
## 4       Scolopendra_viridis_GGNE01.transdecoder.pep_id_GGNE01000196.1.p1
## 5 Hemiscolopendra_marginata_GHBY01.transdecoder.pep_id_GHBY01000549.1.p1
## 6         Ampulex_compressa_GFCP01.transdecoder.pep_id_GFCP01032844.1.p1
##   pfam_query_accession       pfam_name
## 1           PF12698.10 ABC2_membrane_3
## 2           PF12698.10 ABC2_membrane_3
## 3           PF00202.24     Aminotran_3
## 4           PF03098.18   An_peroxidase
## 5           PF03098.18   An_peroxidase
## 6           PF03098.18   An_peroxidase
\end{verbatim}

\begin{Shaded}
\begin{Highlighting}[]
\FunctionTok{colnames}\NormalTok{(additional\_annotations) }\OtherTok{\textless{}{-}} \FunctionTok{c}\NormalTok{(}\StringTok{"hit\_id"}\NormalTok{, }\StringTok{"additional\_pfam\_query\_accession"}\NormalTok{, }\StringTok{"additional\_pfam\_name"}\NormalTok{)}
\NormalTok{toxin\_summaries\_added\_annotations }\OtherTok{\textless{}{-}} \FunctionTok{left\_join}\NormalTok{(toxin\_summaries, additional\_annotations) }\SpecialCharTok{\%\textgreater{}\%} 
  \FunctionTok{filter}\NormalTok{(}\SpecialCharTok{!}\FunctionTok{is.na}\NormalTok{(additional\_pfam\_query\_accession))}
\end{Highlighting}
\end{Shaded}

\begin{verbatim}
## Joining, by = "hit_id"
\end{verbatim}

\begin{Shaded}
\begin{Highlighting}[]
\FunctionTok{head}\NormalTok{(toxin\_summaries\_added\_annotations) }
\end{Highlighting}
\end{Shaded}

\begin{verbatim}
##                                                                   hit_id length
## 1       Boiga_irregularis_GBSH01_id_lcl|GBSH01000763.1_prot_JAG68262.1_1   1479
## 2       Crotalus_horridus_GBKC01_id_lcl|GBKC01000886.1_prot_JAG45184.1_1   1464
## 3         Bothrops_jararaca_GFJM01.transdecoder.pep_id_GFJM01005968.1.p1    826
## 4    Pseudagkistrodon_rudis_GDKV01.transdecoder.pep_id_GDKV01024984.1.p1   2527
## 5 Hemiscolopendra_marginata_GHBY01.transdecoder.pep_id_GHBY01000372.1.p1   1704
## 6    Phoneutria_nigriventer_GFUY01.transdecoder.pep_id_GFUY01025497.1.p1   1758
##   cysteine_content pfam_query_accession pfam_query_name
## 1             0.01           PF00005.30        ABC_tran
## 2             0.01           PF00005.30        ABC_tran
## 3             0.01           PF00664.26    ABC_membrane
## 4             0.02           PF00005.30        ABC_tran
## 5             0.02           PF00005.30        ABC_tran
## 6             0.01            PF13857.9           Ank_5
##   additional_pfam_query_accession additional_pfam_name
## 1                       PF14396.9               CFTR_R
## 2                       PF14396.9               CFTR_R
## 3                       PF14396.9               CFTR_R
## 4                      PF12698.10      ABC2_membrane_3
## 5                      PF12698.10      ABC2_membrane_3
## 6                       PF17809.4                UPA_2
\end{verbatim}

\begin{Shaded}
\begin{Highlighting}[]
\FunctionTok{write.csv}\NormalTok{(toxin\_summaries\_added\_annotations, }\StringTok{"../results/2023{-}04{-}05\_v1/toxin{-}additional{-}domain{-}annotations{-}hits.csv"}\NormalTok{, }\AttributeTok{quote =} \ConstantTok{FALSE}\NormalTok{, }\AttributeTok{row.names =} \ConstantTok{FALSE}\NormalTok{)}
\end{Highlighting}
\end{Shaded}

We can plot the results comparing protein length and cysteine content
while coloring each point either the Pfam accession name for the toxin
or for the additional annotation for proteins that have an additional
annotation:

\begin{Shaded}
\begin{Highlighting}[]
\NormalTok{toxin\_summaries\_added\_annotations }\SpecialCharTok{\%\textgreater{}\%} 
  \FunctionTok{ggplot}\NormalTok{(}\FunctionTok{aes}\NormalTok{(}\AttributeTok{x=}\NormalTok{length, }\AttributeTok{y=}\NormalTok{cysteine\_content)) }\SpecialCharTok{+} 
  \FunctionTok{geom\_point}\NormalTok{(}\FunctionTok{aes}\NormalTok{(}\AttributeTok{color=}\NormalTok{pfam\_query\_name)) }\SpecialCharTok{+}
  \FunctionTok{theme\_classic}\NormalTok{() }\SpecialCharTok{+} 
  \FunctionTok{theme}\NormalTok{(}\AttributeTok{legend.position =} \StringTok{"bottom"}\NormalTok{)}
\end{Highlighting}
\end{Shaded}

\includegraphics{toxin_facilitator_report_files/figure-latex/unnamed-chunk-4-1.pdf}

\begin{Shaded}
\begin{Highlighting}[]
\NormalTok{toxin\_summaries\_added\_annotations }\SpecialCharTok{\%\textgreater{}\%} 
  \FunctionTok{ggplot}\NormalTok{(}\FunctionTok{aes}\NormalTok{(}\AttributeTok{x=}\NormalTok{length, }\AttributeTok{y=}\NormalTok{cysteine\_content)) }\SpecialCharTok{+}
  \FunctionTok{geom\_point}\NormalTok{(}\FunctionTok{aes}\NormalTok{(}\AttributeTok{color=}\NormalTok{additional\_pfam\_name)) }\SpecialCharTok{+} 
  \FunctionTok{theme\_classic}\NormalTok{() }\SpecialCharTok{+}
  \FunctionTok{theme}\NormalTok{(}\AttributeTok{legend.position =} \StringTok{"bottom"}\NormalTok{)}
\end{Highlighting}
\end{Shaded}

\includegraphics{toxin_facilitator_report_files/figure-latex/unnamed-chunk-5-1.pdf}

\begin{Shaded}
\begin{Highlighting}[]
\FunctionTok{sessionInfo}\NormalTok{()}
\end{Highlighting}
\end{Shaded}

\begin{verbatim}
## R version 4.2.1 (2022-06-23)
## Platform: aarch64-apple-darwin20 (64-bit)
## Running under: macOS Monterey 12.5.1
## 
## Matrix products: default
## BLAS:   /Library/Frameworks/R.framework/Versions/4.2-arm64/Resources/lib/libRblas.0.dylib
## LAPACK: /Library/Frameworks/R.framework/Versions/4.2-arm64/Resources/lib/libRlapack.dylib
## 
## locale:
## [1] en_US.UTF-8/en_US.UTF-8/en_US.UTF-8/C/en_US.UTF-8/en_US.UTF-8
## 
## attached base packages:
## [1] stats     graphics  grDevices utils     datasets  methods   base     
## 
## other attached packages:
## [1] forcats_0.5.2   stringr_1.4.1   dplyr_1.0.10    purrr_0.3.5    
## [5] readr_2.1.3     tidyr_1.2.1     tibble_3.1.8    ggplot2_3.4.0  
## [9] tidyverse_1.3.2
## 
## loaded via a namespace (and not attached):
##  [1] tidyselect_1.2.0    xfun_0.35           haven_2.5.1        
##  [4] gargle_1.2.1        colorspace_2.0-3    vctrs_0.5.1        
##  [7] generics_0.1.3      htmltools_0.5.3     yaml_2.3.6         
## [10] utf8_1.2.2          rlang_1.0.6         pillar_1.8.1       
## [13] withr_2.5.0         glue_1.6.2          DBI_1.1.3          
## [16] dbplyr_2.2.1        modelr_0.1.9        readxl_1.4.1       
## [19] lifecycle_1.0.3     munsell_0.5.0       gtable_0.3.1       
## [22] cellranger_1.1.0    rvest_1.0.3         evaluate_0.18      
## [25] labeling_0.4.2      knitr_1.41          tzdb_0.3.0         
## [28] fastmap_1.1.0       fansi_1.0.3         highr_0.9          
## [31] broom_1.0.1         scales_1.2.1        backports_1.4.1    
## [34] googlesheets4_1.0.1 jsonlite_1.8.3      farver_2.1.1       
## [37] fs_1.5.2            hms_1.1.2           digest_0.6.30      
## [40] stringi_1.7.8       grid_4.2.1          cli_3.4.1          
## [43] tools_4.2.1         magrittr_2.0.3      crayon_1.5.2       
## [46] pkgconfig_2.0.3     ellipsis_0.3.2      xml2_1.3.3         
## [49] reprex_2.0.2        googledrive_2.0.0   lubridate_1.8.0    
## [52] assertthat_0.2.1    rmarkdown_2.17      httr_1.4.4         
## [55] rstudioapi_0.14     R6_2.5.1            compiler_4.2.1
\end{verbatim}

\end{document}
